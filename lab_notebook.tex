\documentclass[12pt, a4paper]{article}
\usepackage{graphicx} 
\usepackage{geometry}
\geometry{a4paper, margin=1in}
\usepackage{tikz}
\usetikzlibrary{calc}
\usepackage{hyperref} 
\usepackage{color}
\usepackage{array} % For centering content in table cells
\usepackage{lipsum} % For placeholder text
\usepackage{listings}
\lstset{ 
    language=C,                     % Language of the code
    basicstyle=\ttfamily\footnotesize, % Code font and size
    keywordstyle=\color{blue},      % Keywords color
    stringstyle=\color{red},        % String color
    commentstyle=\color{green},     % Comment color
    morecomment=[l][\color{magenta}]{\#}, % Define more comment style
    numbers=left,                   % Line numbers on the left
    numberstyle=\tiny\color{gray},  % Line numbers style
    stepnumber=1,                   % Line numbers step
    numbersep=10pt,                 % Line numbers separation
    tabsize=4,                      % Tab size
    showspaces=false,               % Do not show spaces
    showstringspaces=false,         % Do not show string spaces
    breaklines=true,                % Automatic line breaking
    breakatwhitespace=false,        % Break at whitespace
    frame=single,                   % Frame around the code
    title=\lstname,                 % Show file name
    escapeinside={\%*}{*)},         % Escape inside code
    morekeywords={*,...}            % Custom keywords
}

% Title, author, and date
\title{
    \vspace{-2cm}
    \Huge \textbf{\color{blue!60} Software Tools And Technology}\\[0.5cm]
    \includegraphics[width=0.3\linewidth]{Makaut.png}\\[0.5cm]
    \LARGE \textbf{\color{black} Lab Notebook}
}
\author{
    \vspace{1cm}
    \Large Group 22
}
\date{} % Leave empty to manually specify the date

\begin{document}
\maketitle
\pagenumbering{gobble}

% Border
\begin{tikzpicture}
    [remember picture, overlay]
    \draw[line width = 2pt, blue] 
        ($(current page.north west) + (1cm,-1cm)$) 
        rectangle 
        ($(current page.south east) + (-1cm,1cm)$);
\end{tikzpicture}

\vspace{-1cm}
\begin{center}
    \textbf{Repository Link:} \href{https://github.com/Sayantan2005/Group\_22}{\textcolor{blue!60}{https://github.com/Sayantan2005/Group\_22}}
\end{center}

\vspace{1cm}

\centering
\bfseries{\underline{\Large \textcolor{blue!60}{Group Members:}}}
\vspace{0.5cm}

\begin{flushleft}
\begin{enumerate}
    \item \textbf{Leader Name: Sayantan Sarkar}\\
    \textit{Roll No: 30085323013} \\
    Department: BSc in IT (Cyber Security)
    
    \item \textbf{Collaborator 1: Diya Sarkar} \\
    \textit{Roll No: 30059223032} \\
    Department: BSc in Forensic Science
    
    \item \textbf{Collaborator 2: Shinjinee Biswas} \\
    \textit{Roll No: 30001223029} \\
    Department: BCA
    
    \item \textbf{Collaborator 3: Sampurna Das} \\
    \textit{Roll No: 30059223038} \\
    Department: BSc in Forensic Science
    
    \item \textbf{Collaborator 4: Sidhita Dey} \\
    \textit{Roll No: 30059223020} \\
    Department: BSc in Forensic Science
\end{enumerate}
\end{flushleft}

\vspace{1cm}
\begin{center}
\textbf{Instructor:} \textcolor{blue!60}{Ayan Ghosh} \\
\vspace{0.3cm}
\textit{Date: \today}
\end{center}

\newpage
\begin{tikzpicture}
    [remember picture, overlay]
    \draw[line width = 2pt, blue] 
        ($(current page.north west) + (1cm,-1cm)$) 
        rectangle 
        ($(current page.south east) + (-1cm,1cm)$);
\end{tikzpicture}

% Index Table
\section*{\underline{\Huge\textbf{\textcolor{blue!60}{Index Table}}}}
\vspace{0.5cm}

\renewcommand{\arraystretch}{3} % Adjusts row height
\setlength{\tabcolsep}{20pt} % Adjusts column padding

\begin{tabular}{|>{\centering\arraybackslash}p{80pt}|>{\centering\arraybackslash}p{250pt}|}
\hline
\textbf{Serial No.} & \textbf{Questions} \\
\hline
1 & Introduction to Github and Github desktop version installation \\\hline
2 & Building a C programme of calculator in the local repository, committing and publishing it as a public repository \\\hline
3 & Converting submit button to Chin tapak dum dum \\\hline
% 4 & How do you analyze the results? \\\hline
% 5 & What conclusions can be drawn from the results? \\\hline
% 6 & What references were used for this lab? \\
% \hline
\end{tabular}



\newpage
\begin{tikzpicture}
    [remember picture, overlay]
    \draw[line width = 2pt, blue] 
        ($(current page.north west) + (1cm,-1cm)$) 
        rectangle 
        ($(current page.south east) + (-1cm,1cm)$);
\end{tikzpicture}
\vspace{-2cm}
% Lab notebook entries
\section*{\Huge{\textcolor{blue!60}{Lab Notebook Entries}}}

% Each member writes their entry below
\subsection*{Entry by Sayantan Sarkar}
\textit{Date: [\today]}\\

% Insert the image below the date and above the GitHub section
\begin{figure}[h!]
   \centering
    \includegraphics[width=0.5\linewidth]{Github.png}
\end{figure}

\vspace{-1cm} % Adjust this value if you want more space between the image and the text below

\section*{\Huge{GitHub}}
\paragraph{GitHub is a web-based platform that allows developers to host, share, and collaborate on software projects. It provides a version control system powered by Git, enabling teams to track changes, manage code repositories, and work together seamlessly, even across different locations. GitHub supports collaborative development through features like pull requests, issues, and project boards, making it an essential tool for open-source projects and professional software development alike. Additionally, it offers integration with various development tools, enhancing productivity and streamlining the software development lifecycle.}

\subsection*{Installation}
\paragraph{Installing GitHub Desktop is a straightforward process that enhances your workflow by providing a user-friendly interface for managing repositories. To begin, download the installer from the [official GitHub Desktop website](https://desktop.github.com/) for your operating system—Windows or macOS. After downloading, simply run the installer and follow the on-screen instructions to complete the setup. Once installed, you can launch the application and sign in with your GitHub credentials, or create a new account if needed. GitHub Desktop streamlines the process of cloning repositories, making commits, and managing branches, making it an invaluable tool for developers of all skill levels. For Linux users, alternative methods like using Wine or other Git clients are available.}


\newpage
\begin{tikzpicture}
    [remember picture, overlay]
    \draw[line width = 2pt, blue] 
        ($(current page.north west) + (1cm,-1cm)$) 
        rectangle 
        ($(current page.south east) + (-1cm,1cm)$);
\end{tikzpicture}
\vspace{-2cm}
\subsection*{Entry by Diya Sarkar}
\textit{Date: [\today]}\\
\section*{\LARGE{Build a c program of calculator in the local repository, commit and publish it as a {\underline{-:public repository:-}}}}


\vspace{2 cm}
\Large{Below is a basic C program for a calculator that can perform addition, subtraction, multiplication, and division:}
\vspace{0.5 cm}


\begin{lstlisting}[language=C, basicstyle=\ttfamily\footnotesize, keywordstyle=\color{blue}]
#include <stdio.h>

int main() {
    char operator;
    double num1, num2;
    printf("Enter an operator (+, -, *, /): ");
    scanf("%c", &operator);
    printf("Enter two operands: ");
    scanf("%lf %lf", &num1, &num2);

    switch (operator) {
        case '+':
            printf("%.2lf + %.2lf = %.2lf\n", num1, num2, num1 + num2);
            break;
        case '-':
            printf("%.2lf - %.2lf = %.2lf\n", num1, num2, num1 - num2);
            break;
        case '*':
            printf("%.2lf * %.2lf = %.2lf\n", num1, num2, num1 * num2);
            break;
        case '/':
            if (num2 != 0.0)
                printf("%.2lf / %.2lf = %.2lf\n", num1, num2, num1 / num2);
            else
                printf("Error! Division by zero.\n");
            break;
        default:
            printf("Invalid operator!\n");
            break;
    }
    return 0;
}
\end{lstlisting}
\newpage
\begin{tikzpicture}
    [remember picture, overlay]
    \draw[line width = 2pt, blue] 
        ($(current page.north west) + (1cm,-1cm)$) 
        rectangle 
        ($(current page.south east) + (-1cm,1cm)$);
\end{tikzpicture}
\vspace{-2 cm}
\section*{Step 2: Create a Local Git Repository and Commit the Program}

\begin{enumerate}
    \item Open your terminal or command prompt.
    \item Navigate to the directory where your \texttt{calculator.c} file is located:
    
    \begin{lstlisting}[language=bash, basicstyle=\ttfamily\footnotesize]
cd /path/to/your/directory
    \end{lstlisting}
    
    \item Initialize a Git repository in the directory:
    
    \begin{lstlisting}[language=bash, basicstyle=\ttfamily\footnotesize]
git init
    \end{lstlisting}
    
    \item Add the \texttt{calculator.c} file to the staging area:
    
    \begin{lstlisting}[language=bash, basicstyle=\ttfamily\footnotesize]
git add calculator.c
    \end{lstlisting}
    
    \item Commit the changes:
    
    \begin{lstlisting}[language=bash, basicstyle=\ttfamily\footnotesize]
git commit -m "Initial commit: Added calculator program"
    \end{lstlisting}
\end{enumerate}

\section*{Step 3: Publish the Repository to GitHub}

\begin{small}
\begin{enumerate}
    \item Create a new repository on \href{https://github.com}{GitHub}. Make it public, and don't initialize it with any files.
    \item Copy the repository URL (e.g., \texttt{https://github.com/yourusername/repositoryname.git})
    \item In your terminal, add the GitHub repository as a remote:
    
    \begin{lstlisting}[language=bash, basicstyle=\ttfamily\footnotesize]
git remote add origin https://github.com/diya-20/Diya_Cal.git
    \end{lstlisting}
    
    \item Push your local repository to GitHub:
    
    \begin{lstlisting}[language=bash, basicstyle=\ttfamily\footnotesize]
git push -u origin master
    \end{lstlisting}
\end{enumerate}
\end{small}

\newpage
\begin{tikzpicture}
    [remember picture, overlay]
    \draw[line width = 2pt, blue] 
        ($(current page.north west) + (1cm,-1cm)$) 
        rectangle 
        ($(current page.south east) + (-1cm,1cm)$);
\end{tikzpicture}
\vspace{-2cm}
\subsection*{Entry by Sampurna Das}
\textit{Date: [\today]}\\
\section*{Java Swing Application: SymbolApp}
This document describes the Java Swing application named \texttt{SymbolApp}. The application showcases a simple "mind-reading" trick by displaying a grid of symbols and revealing a selected symbol based on user interaction. This document is formatted using LaTeX to provide a clear and professional presentation for academic purposes.

\section*{ 1. Clone the Repository}
At first, I used GitHub Desktop to clone the repository: https://github.com/GeekAyan/STT.
Then I Opened GitHub Desktop, clicked on "File" then "Clone Repository", pasted the URL, and selected my local directory.

\section*{  2. Set Up the Project}
I opened the project in VSCode.
Then Followed the detailed run instructions provided in the README.md file to set up any necessary dependencies and configurations. This could involve installing Python packages or setting environment variables.

\section*{ 3. Run the Application}
I Run the application according to the instructions to ensure everything is working as expected.

\section*{ 4. Modify the Button}
I located the code for the button in the project files. This could be in a JavaScript, HTML, or Python file, depending on the technology stack used.
I Renamed the button text to \textbf{Chin Tapak Dum Dum"}.
  
\section*{ 5. Fix the Button Proportions}
After renaming the button, I analyzed why the button looks disproportionate. Possible fixes could involve. I adjusted something and modified the code.
\newpage
\begin{tikzpicture}
    [remember picture, overlay]
    \draw[line width = 2pt, blue] 
        ($(current page.north west) + (1cm,-1cm)$) 
        rectangle 
        ($(current page.south east) + (-1cm,1cm)$);
\end{tikzpicture}
\vspace{-2cm}
\section*{ 6. Test the Changes}
I ran the application again to ensure the button now appears correctly proportioned and that it functioned as intended.

\section*{ 7. Commit the Changes}
I saved all the changes in your IDE.
In GitHub Desktop,I commited the changes with a descriptive message like “Fixed button proportions and renamed to 'Chin Tapak Dum Dum'”.

\section*{ 8. Push Changes to my Fork}
I pushed the changes to my forked version first.

\section*{9. Create a Pull Request}
I went to the original GitHub repository on my web browser.
Clicked on “Pull Requests” then “New Pull Request”.
Compare my branch with the main branch of the original repository.
Added a title and description explaining my changes and why they were made.
\section*{Code Description}
The \texttt{SymbolApp} class extends \texttt{Frame} and implements \texttt{ActionListener}. It generates a random symbol and displays it among other symbols in a grid layout. The user follows specific steps to select a symbol, which is then revealed when the submit button is clicked.

\subsection{Code Listing}
\begin{lstlisting}[language=Java, caption=Java Swing Application Code]
import java.awt.*;
import java.awt.event.*;
import java.util.Random;

public class SymbolApp extends Frame implements ActionListener {
    private Label[] symbolLabels = new Label[99];
    private Button submitButton;
    private String specialSymbol;
    private String selectedSymbol;
    public SymbolApp() {
        // Generate a random special symbol
        Random rand = new Random();
        specialSymbol = Character.toString((char) (rand.nextInt(94)
        + 33)); // Random ASCII character from 33 to 126
        selectedSymbol = specialSymbol;

        // Setting up the main frame
        setLayout(new BorderLayout());
        setSize(800, 700);
        setTitle("Symbol App");

        // Adding instruction message
        TextArea instruction = new TextArea(
            "Think of any two digit number. Now reverse it and find
            the difference of them.\n" +
            "Now find the number you got and remember the symbol from
            the panel below.\n" +
            "Don't tell me, I'll read your mind!
            Hit the below button when you are ready
            to see the magic!",
            5, 60, TextArea.SCROLLBARS_NONE);
        instruction.setEditable(false);
        instruction.setFont(new Font("Arial", Font.PLAIN, 16));
        add(instruction, BorderLayout.NORTH);

        // Panel for symbols
        Panel symbolPanel = new Panel(new GridLayout(11, 9));
        for (int i = 0; i < 99; i++) {
            String symbol = (i % 9 == 0) ? specialSymbol : 
            Character.toString
            ((char) 
            (33 + (i % 94)));
            symbolLabels[i] = new Label(i + ": " + symbol);
            symbolLabels[i].setAlignment(Label.CENTER);
            symbolPanel.add(symbolLabels[i]);
        }
        add(symbolPanel, BorderLayout.CENTER);

        // Panel for submit button
        Panel controlPanel = new Panel(new FlowLayout());
        // changed button name to chin tapak dum dum
        submitButton = new Button("Chin Tapak Dum Dum"); 
        submitButton.addActionListener(this);
        controlPanel.add(submitButton);
        add(controlPanel, BorderLayout.SOUTH);

        // Setting up the window close event
        addWindowListener(new WindowAdapter() {
            public void windowClosing(WindowEvent we) {
                System.exit(0);
            }
        });

        setVisible(true);
    }
\end{lstlisting}


\newpage
\begin{tikzpicture}
    [remember picture, overlay]
    \draw[line width = 2pt, blue] 
        ($(current page.north west) + (1cm,-1cm)$) 
        rectangle 
        ($(current page.south east) + (-1cm,1cm)$);
\end{tikzpicture}
\vspace{-2cm}
\subsection*{Entry by Shinjinee Biswas}
\textit{Date: [\today]}\\
% Write your lab notebook entry here.

\newpage
\begin{tikzpicture}
    [remember picture, overlay]
    \draw[line width = 2pt, blue] 
        ($(current page.north west) + (1cm,-1cm)$) 
        rectangle 
        ($(current page.south east) + (-1cm,1cm)$);
\end{tikzpicture}
\vspace{-2cm}
\subsection*{Entry by Sidhita Dey}
\textit{Date: [\today]}\\
% Write your lab notebook entry here.


\end{document}
